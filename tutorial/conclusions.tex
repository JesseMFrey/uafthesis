\chapter{Conclusions}
\label{sec:conclusions}

These conclusions are in two sections to address the two most
frequently asked questions about my research: ``What are you
studying?'' and ``What is that good for?''  The second question,
sadly, has not just been asked by non-scientists.  The long answer to
the first question is contained in previous chapters, and I will
recapitulate the specific results of that research in this next
section.  In the second section, I will discuss my view of the second
question, using it as an opportunity to distill some of the general
knowledge of complex systems research that I have learned over the
last few years.

\section{Summary of Results}
\label{sec:summary-results}

\subsection{Long Time Correlations Exist in Very Weakly
  Driven SOC Systems}
\label{sec:long-time-corr}

I feel that the most important results of this thesis are contained in
Chapter \ref{sec:partI}.  The results overturn ideas that have been
accepted since 1992.  For a wholly SOC system there is no minimum
driving rate necessary for it to remain SOC.  Once the first grain of
sand falls and increases a local gradient, there will {\em always} be
a higher probability of a relaxation event occurring at that location
in the future than if the grain had not fallen.  This increase in a
single gradient that persists is the essence, the building block, of a
correlation.  If much time passes before the next fluctuation finally
triggers an avalanche, that is a long time correlation.  There are
many connected sites in the complex systems that a sandpile represents
and the totality of all of the fluctuations captured in a long enough
time series reflects the correlations through rescaled range (\rs)
analysis and the power spectrum.  Scaling exponents of these two
measures are nontrivial for time scales much longer than previously
thought.  Here, nontrivial means that the Hurst exponent, the slope of
the \rsa, is not 0.5 and that the slope of the power spectrum, \bb\,
is not 0.  If $\hh = 0.5$ and $\bb = 0$ then a time series is
considered random without correlations.

\begin{table*}
\caption[The Big Analogy of Sandpiles.]{The Big Analogy of Sandpiles.
  Can you think of more?}  \vspace{12pt} 
  \begin{tabular}{|>{\PBS\centering\hspace{0pt}}m{.8in}%
      ||>{\PBS\centering\hspace{0pt}}m{.9in}%
      |>{\PBS\centering\hspace{0pt}}m{.9in}%
      |>{\PBS\centering\hspace{0pt}}m{1.5in}|}\hline
    parameter &  
    sandpile & 
    tectonics &
    space plasma \\
    \hline \hline
    \centering \LL & 
    size & 
    fault & 
    magnetospheric scales 
    \\ \hline
    \po & 
    external drive (sand) & 
    slip rate & 
    solar wind fluctuations 
    \\ \hline
    \zc & 
    threshold & 
    critical stress & 
    plasma pressure, current gradients
    \\ \hline
    \nf & 
    response & 
    local reduc. of stress & 
    plasma/MHD instabilities \\
    \hline \hline 
    event &
    avalanche &
    earthquake &
    burst of plasma \\
    \hline
  \end{tabular}
  \centering
  \label{tab:fourparams}
\end{table*}

\section{The Worth of Sandpiles}
\label{sec:worth-sandpiles}

What is this research good for?  Better yet, why is this research
exciting?  This section is my reward for finishing a degree.  This is
the fun part, to be able to write about why this is interesting to me
and to give a little discourse to anyone reading this who wonders
about this branch of science.  After all, the degree towards which I
have been working is a Doctor of {\em Philosophy}.  Also, though, this
section is a reply to my interested but doubting, questioning self of
four years ago when I wondered about the worth of sandpiles.

In SOC, correlations among events are measured statistically and
dynamically not specifically.  So SOC does not predict when the next
event happens given the current state of affairs.  But it does predict
the long time statistical and dynamical behavior of many such events
through the signatures of the PDFs, spectra and \rs.  Some people want
specific predictions.  When will the ball land?  How big should the
bridge be?  How fast can I download those images?  But people must
come to accept that not all predictions are specific.  Insurance
companies know and accept this and go on their merry way to the bank.
Science does not always say exactly where and when.  Heisenberg said
that.  Simple models can help explain how and why.  I said that.

