\chapter{Building Blocks of Self-Organized Criticality, Part II: \\
  Transition from Very Low Drive to High Drive}
\label{sec:partII}

\bigskip

\section*{Abstract}
\label{sec:partII_abstract}

We analyze the transition of the self-organized criticality one
dimensional directed running sandpile model of Hwa and Kardar [Phys.
Rev. A {\bf 45}, 7002 (1992)] from very low external forcing to high
forcing, showing how six distinct power law regions in the power
spectrum at low drive become four regions at high drive.  One of these
regions is due to long time correlations among events in the system
with steady state power spectrum that scales as $\sim f^{- \bb}$ with
$0 < \bb \leq 1$.  The location in frequency space and the value of
\bb\ both increase as the external forcing increases.  \bb\ ranges
from $\approx 0.4$ for the weakest forcing studied here to a maximum
value of 1 (i.e., a \ff\ region) at stronger levels.  The change from
low to high \bb\ occurs when the average quiet time between avalanche
events is on the same order as the average duration of events.  The
correlations are quantified by a constant Hurst exponent $\hh \approx
0.8$ when estimated by \rsa\ for sandpile driving rates spanning over
five orders of magnitude.  The constant \hh\ and changing \bb\ in the
same system as forcing changes suggests that the power spectrum does
not consistently quantify long time dynamical correlations and that
the relation \bbh\ does not hold for the time series produced by this
SOC model.  Because of the constant rules of the model we show that
the same physics that produces a $\bb = 1$ scaling region during
strong forcing produces a $0 < \bb < 1$ region at weaker forcing.

\begin{table*}
\caption[Time period, resolution, slopes of first two spectral regions
  and breakpoint of AE index data.]{Time period, resolution, slopes of
  first two spectral regions and breakpoint of AE index data.  Data
  found in \cite{tsurutani90a}, 
  \cite{consolini96a}, \cite{uritsky98a}, \cite{price01a} and
  \cite{watkins02a}.   Breakpoint for \cite{consolini96a} taken between
  labeled second and third regions.  Breakpoint for \cite{price01a}
  1978-1979 estimated from plot at intersection of two power law fits.
  Slope for \cite{watkins02a} taken as best fit
  with a straight edge, slope estimated from axes.}  \vspace{12pt}

\begin{tabular}{c | c | c | c | c | c}
Study & Period & Res. & $\bb_A$ & $\bb_B$ & Break (mHz) \\
\hline
\cite{tsurutani90a} & 1967--1970 & 5 min & 2.42 & 1.02 & 0.059 (4.7 hr) \\
\cite{tsurutani90a} & 1971--1974 & 1 hr  & 2.2  & 0.98 & 0.050 (5.5 hr) \\
\cite{uritsky98a} & 1973--1974 & 1 hr & 2.10 & 0.95 & 0.056 (5.0 hr) \\
\cite{consolini96a} & 1/1--19/2 1975 & 1 min & 2.65 & 1.14 & 0.073 (3.8 hr) \\
\cite{watkins02a} & 1978 & 5 min & 2.1 & 1.1 & 0.056 (5.0 hr) \\
\cite{price01a} & 1978--1979  & 1 min & 1.85 & 0.82 & 0.033 (8.4 hr) \\
\cite{tsurutani90a} & 1978--1980 & 1 hr  & 2.2  & 1.00 & 0.056 (5.0 hr) \\
\cite{price01a} & March 1979 & 1 min & 1.89 & n/a   &  n/a \\
& & & & & \\
Mean $\pm$ $\sigma$ & & & $2.4 \pm 0.26$ & $1.0 \pm 0.10$ & \\
\end{tabular}
\centering
\label{tab:aeslopebreakpoint}
\end{table*}

\section{Introduction}
\label{sec:partII_introduction}

Simple models have been used to study the dynamics of many physical
systems, such as confined fusion plasmas \cite{newman96a,newman96b},
space plasmas \cite{lui00a,chapman01a} and earthquakes
\cite{turcotte1997a}, among others.  These models comprise a connected
network of local nonlinear gradients that can persist because of a
critical threshold.  Random external forcing of the system increases
local gradients; when one of them exceeds the critical threshold a
relaxation event is triggered that stabilizes the gradient.  The
gradient is reduced by transferring mass, heat, stress or some other
quantity specific to the system to neighboring regions which can make
them unstable, creating a series of relaxations.  This sequence of
events, called an avalanche, occurs much faster than the external
drive increases the gradients.  These models and this type of dynamics
are characteristic of self-organized criticality (SOC)
\cite{btw87a,btw88a,bak1996a}.

One of the first SOC models was the sandpile
\cite{btw88a,kadanoff89a,manna1991a}.  A one dimensional variation of
it was studied for strong external forcing by \cite{hwakardar92a} and
later for weak external forcing by \cite{woodard04a}.  Both studies
show that even though the system is randomly driven, long time
correlations exist in the dynamics on time scales much longer than the
duration of any single avalanche.  The question of whether long time
dynamical correlations exist in a time series---a basis for
predictability \cite{yang04a,woodard04f}---is fundamental to many
physical and geophysical fields.

%% cg100_power_L0200_PII
\begin{figure}
  \centering
  \includegraphics[height=\onepicht]{\figeps{gnu-head-sm}}
  \caption[This first line must be the same in List and caption.]{
    This first line must be the same in List and caption.  But then
    you can have a bunch of other stuff.  Power spectra of flips time
    series of $\LL = 200$ sandpile for five orders of magnitude of
    effective driving rate in $\poll \in (0.002, 296)$.  Spectra have
    been shifted along $y$ axis for easier viewing.}
  \label{fig:merged_all_stretched}
\end{figure}

\section{Model and Methods}
\label{sec:partII_one-dimens-runn}

We analyze the flips time series with the power spectrum and \rsa.
For a data series $X(t)$, the power spectrum is defined as $S(f) =
\left|F(f)\right|^2$, where $F(f)$ is the power spectrum.  The
rescaled range \cite{hurst51a,mandel2002a} is defined as $R'(\tau)
\equiv R(\tau)/S(\tau)$, where $S(\tau)$ is the standard deviation and
\begin{gather*}
  R(\tau) = \underset{1 \leq k \leq \tau}{\max}W(k,\tau) - \underset{1 \leq k
    \leq \tau}{\min}W(k,\tau) \; \; \; \mathrm{(range),} \\
  W(k,\tau) = \sum_{t=1}^{k} (X_t - \langle X \rangle_\tau) \; \;
  \mathrm{(cumulative \; \; deviation) \; \; and} \\
  \langle X \rangle_\tau = \frac{1}{\tau} \sum_{t=1}^{\tau} X_t \; \; \;
  \mathrm{(mean).}
\end{gather*}
If the rescaled range of the time series scales as $R'(\tau) \sim
\tau^H$, the slope of the plot of $R'(\tau)$ versus the time lag
$\tau$ on a doubly logarithmic plot is the Hurst exponent, $H$.  

\section{Results}
\label{sec:partII_results}

Figure \ref{fig:merged_all_stretched} shows the power spectra of the
flips time series of the one dimensional directed running sandpile for
over five orders of magnitude of effective driving rate, \poll, which
increases from top to bottom in the figure.  The sandpile size is $\LL
= 200$. We have studied sandpile sizes up to $\LL = 2000$ and found
the behavior to be consistent with that of the smaller system.  The
lowest drive used is $\poll = 0.002$ and the highest is $\poll = 296$.
The higher limit is chosen to stay below the normal overdrive limit of
$\pol < \nf / 2$ (derived in Section \ref{sec:partII_conclusions}).

Three spectra from Figure \ref{fig:merged_all_stretched} are shown in
Figure \ref{fig:powerrs}(a), representing low, medium and high driving
rates of the sandpile.  The six regions of low drive and four regions
of high drive are shown by the solid lines.  The lines are power laws
\fb\ and the numbers next to them are the values of \bb.  The lowest
frequency \fo\ region of the low drive case is not seen because of the
finite size of the time series.  Its existence is assumed based on the
\fo\ regions seen in the spectra of higher drive cases.

\begin{figure*}
  \centering 
  \begin{tabular}{c@{\hspace{25pt}}c}
    \includegraphics[height=\twopichtside]{\figeps{penguin}} &
    \includegraphics[height=\twopichtside]{\figeps{penguinuaf}} \\
  \end{tabular}
  \caption[(a) Power spectra and (b) \rsa\ of flips for
  different driving rates.]  {(a) Power spectra and (b) \rsa\ of flips
    for different driving rates.  The $y$ values of both measures have
    been shifted for easier viewing.  Numbers shown are the exponents
    of power law fits to regions, \bb\ for the spectra and \hh\ for
    \rs.}
  \label{fig:powerrs}
\end{figure*}

The associated \rsa\ for the low, medium and high drive power spectra
are shown in Figure \ref{fig:powerrs}(b).  Five regions at low drive
become four regions at high drive.  Power law lines and their slopes
are indicated in the figure.  The slopes are the Hurst exponent \hh\ 
for each region.  Again, the region for the longest time scales at the
lowest drive is not seen because of the finite length of the time
series but is inferred based on the higher drive cases.

The breakpoints between regions in the two different measures, power
spectrum and \rs, can be compared with each other.  The breakpoints of
the two measures, found independently, agree very closely with each
other, though the \rs\ breakpoints appear at slightly longer time
scales than those of the power spectrum.  This effect is known from
comparisons of \rsa\ with structure functions \cite{gilmore02a} and we
conclude that both measures can distinguish the same dynamical regions
through the identification of different power law regions.

\section{Conclusions}
\label{sec:partII_conclusions}

We have analyzed the one dimensional directed running sandpile SOC
model for five orders of magnitude of effective driving rate and for
different system sizes and shown how the power spectrum and \rsa\ 
change from low drive to high drive.  The most noticeable feature of
the change in signatures is the loss of the power law region C at low
drive with $\bb = 0$ and $\hh = 0.5$.  This region is due to
uncorrelated quiet times between distinct individual avalanches.  The
region disappears because events are triggered more frequently in the
sandpile as driving rate increases; this causes a virtual extinction
of quiet times.  \bb\ and \hh\ of this uncorrelated region increase
with driving rate until they reach limiting values of $\bb \approx 1$
and $\hh \approx 0.8$, both being signs of long time correlations.
The greatest change in \bb\ with increasing driving rate is when the
average quiet time is on the order of the average avalanche duration.

\begin{figure*}
  \centering 
  \begin{tabular}{c@{\hspace{25pt}}c}
    \includegraphics[height=\twopichtside]{\figeps{cartoon_power_low_PII}} &
    \includegraphics[height=\twopichtside]{\figeps{cartoon_rs_low_PII}} \\
    \includegraphics[height=\twopichtside]{\figeps{cartoon_power_high_PII}} &
    \includegraphics[height=\twopichtside]{\figeps{cartoon_rs_high_PII}} \\
  \end{tabular}
  \caption[Cartoons of distinct regions and their breakpoints and
  causes of power spectra and \rsa\ of sandpile flips.]{Cartoons of
    distinct regions and their breakpoints and causes of power spectra
    and \rsa\ of sandpile flips.  (a) Power spectrum of low drive, (b)
    \rsa\ of low drive, (c) power spectrum of high drive and (d) \rsa\ 
    of high drive.  (c) is taken from Figure 6 of \cite{hwakardar92a}
    and the others are drawn in that spirit.  (a) and (b) are from
    \cite{woodard04a} but are reproduced here for completeness.}
  \label{fig:cartoons}
\end{figure*}

