\chapter{Building Blocks of Self-Organized Criticality, Part I: \\
The Very Low Drive Case}
\label{sec:partI}

\bigskip

\section*{Abstract}
\label{sec:partI_abstract}
We describe new analyses and signatures of the self-organized critical
one dimensional directed running sandpile model of Hwa and Kardar
[Phys. Rev. A {\bf 45}, 7002 (1992)].  We present results for
extremely low levels of external forcing of this SOC model and show
that correlations in the dynamics exist over very long time scales
regardless of how low this driving rate is.  This demonstrates that a
SOC system has nontrivial dynamics even when the system's events do
not overlap in space or time.  A consequence of this is that the power
spectral and rescaled range ($R/S$) analysis signatures of the SOC
time series for very weak forcing are very different from a simple
random superposition of pulses.

\section{Introduction}
\label{sec:partI_introduction}

Self-organized criticality (SOC) \citet{btw87a,btw88a} is a dynamical
framework that describes how certain large-scale complex behavior can
emerge from a system of small-scale simple interactions.  SOC concerns
the dynamics of nonequilibrium systems that have a local critical
threshold.  If this threshold is constant throughout the entire
system, then an average constant global gradient is maintained through
two opposing mechanisms: an external forcing that increases the
gradient and internal transport of the quantity that reduces the
gradient.  The relaxation of the gradient usually occurs in a series
of aperiodic bursts, called avalanches in SOC lingo.  The avalanche
mechanism allows for stable gradients to exist in the system and
contrasts with linear diffusion, which constantly acts to reduce any
gradient.  In SOC, the avalanches take place on time scales that are
much shorter than those of the external forcing.

\section{Model}
\label{sec:partI_one-dimens-runn}

The prototypical SOC model is known as the sandpile.  The name was
chosen to produce a good, simple mental picture, not because it
necessarily models real sandpiles.  There are many varieties of
sandpile models \citet{btw88a,manna1991a,jensen98a}; we will only
describe the one dimensional directed running sandpile of
\citet{hwakardar92a}. In addition to general SOC theory, this model is
useful in studying physical systems where the dynamics can be reduced
to a one dimensional approximation.  One example of this is a fusion
plasma confined in a tokamak \citet{diamond95a,newman96a}, where,
because of toroidal and poloidal symmetries, plasma transport can be
approximated by a steady gradient in one dimension.  The single
dimension can represent gradient-driven turbulent transport of plasma,
heat and density from the hot dense core to the cooler, less dense
edge of the tokamak.

\section{Methods}
\label{sec:partI_methods}

This model is a dynamical system; a characteristic of such systems
that can quantify the dynamics is long time correlations.  We study
long time correlations with the power spectrum and rescaled range
($R/S$) analysis.  The power spectrum is defined as the square of the
Fourier transform, $S(f) = \left|F(f)\right|^2$, where $F(f) = N^{-1}
\sum_{t = 0}^{N - 1} X(t) e^{-i 2 \pi (f / N) t}$.  For a finite real
time series, the spectrum also equals the Fourier transform of the
autocorrelation function of the time series.

\section{Results}
\label{sec:partI_results:-low-drive}

The study reported here is of the SOC sandpile model for very low
external forcing.  Very low drive can be defined empirically by
examining the flips time series and choosing cases where individual
avalanches are distinct and well-separated by quiet times, that is,
where there is no overlapping of events.  

\subsection{Region D:  SOC and Correlated Events}
\label{sec:partI_region-d:-soc}

The physics of this region is the main point of this paper.  Region D
is the only true dynamical SOC region in the sense that its signatures
arise solely from interactions and correlations among separate events
in the system.  {\em On the time scales in this region, the signatures
  reflect only long time correlations and nothing about pulse shape,
  quiet times, random superpositions, overlapping of pulses or system
  size.}  Because the high frequency end of this region is due to
driving rate and the low frequency end is due to the finite capacity
of the sandpile, larger systems have larger regions D.  The limiting
extension is that a system of infinite size would have a region D that
extends to infinitely low frequencies and there would be no regions E
or F.

\section{Conclusions}
\label{sec:partI_conclusions}

We have studied the one dimensional directed running sandpile at very
low drive and have shown that correlations from a memory mechanism in
SOC dynamics produce nontrivial signatures in the power spectra and
$R/S$ analysis of flips time series.  The memory is stored in the
local gradient of each cell, regardless of driving rate.  The
signatures of the correlations appear at longer time scales for lower
external forcing.  A consequence of this is that a time series for any
system, be it a defined SOC model or a real physical system suspected
of being SOC, can be too short to see the correlation signatures and
thus could be mistaken for a simple random time series.  Given long
enough time series, a very distinct difference can be seen between the
signatures of random data and the sandpile data.  The sandpile chooses
the particular size, order and separation of events in a way that is
very different from any random combination of size, order and
separation.

